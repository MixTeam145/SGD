\documentclass[a4paper]{article}
\usepackage[utf8]{inputenc}
\usepackage[english,russian]{babel}

\usepackage{geometry}

\usepackage{amsmath}
\usepackage{amssymb}
\usepackage{amsthm}
\usepackage{bm}

\usepackage{array}
\usepackage[position=below]{subfig}


\title{Применение стохастического градиентного спуска в обобщенных линейных моделях}
\date{}

\begin{document}
\maketitle

% \section{Постановка задачи}
% Дана зависимая переменная $\mathbf{y}$ и матрица данных $\mathbf{X}$. Предположим, что зависимая переменная имеет распределение, входящее в экспоненциальное семейство. Модель:
% \[
% \mathsf{E}(\mathbf{y}~|~\mathbf{X})=\mathsf{\boldsymbol{\mu}}=g^{-1}\left(\mathbf{X}\boldsymbol{\beta}\right),
% \]
% где $\boldsymbol{\beta}$ --- вектор параметров, $g$ --- линк-функция. Задача: найти такой вектор параметров $\hat{\boldsymbol{\beta}}$, что
% \[
% \hat{\boldsymbol{\beta}}=\operatorname*{argmax}_{\boldsymbol{\beta}}\sum_{i=1}^n \ln p(y_i;\boldsymbol{\beta}),
% \]
% где $p(y_i; \boldsymbol{\beta})$ --- функция правдоподобия в точке $y_i$.

\section{Теоретическая часть}
\subsection{Линейная регрессия}
Модель:
\[
    \mathbf{y}\sim N(0, \sigma^2),\quad \mathsf{E}(\mathbf{y}~|~\mathbf{X})=\bm\mu=\mathbf{X} \bm\beta,
\]
где $\mathbf{X}$ — матрица данных, $\boldsymbol\beta$ — вектор параметров. Плотность нормального распределения:
\[
    p(y; \mu, \sigma^2)=\frac1{\sqrt{2\pi\sigma^2}}\exp\left\{-\frac{(x-\mu)^2}{2\sigma^2}\right\},\quad y\in \mathbb{R}.
\]
Матожидание и дисперсия: $\mathsf E \xi = \mu$, $\mathsf D \xi=\sigma^2=\phi$. Логарифм правдоподобия:
\[
    L(\mathbf{y}; \mu, \phi)=-\frac{n}{2}\ln2\pi\phi-\frac{1}{2\phi}\sum_{i=1}^n(y_i-\mu)^2.
\]
Тогда, поскольку параметр $\phi$, вообще говоря, неизвестен, необходимо оптимизировать функцию потерь и по $\phi$. Производные по вектору параметров $\bm\beta$ и $\phi$:
\[
    \frac{\partial L}{\partial\bm\beta}=\frac1\phi\mathbf{X}^\mathrm{T}(\mathbf{y}-\bm\mu),\quad \frac{\partial L}{\partial \phi}=-\frac{n}{2\phi} + \frac{1}{2\phi^2}\sum_{i=1}^n(y_i-\mu_i)^2.
\]

\subsection{Гамма-регрессия}
Модель:
\[
    \mathbf{y}\sim\Gamma(k, \theta), \quad \mathsf{E}(\mathbf y~|~ \mathbf{X})=\bm\mu=g^{-1}(\mathbf{X}\bm{\beta}),
\]
где $\mathbf{X}$ — матрица данных, $\boldsymbol\beta$ — вектор параметров, $g$ --- линк-функция.
Плотность гамма-распределения:
\[
    p(y; k, \theta) = \frac{y^{k-1}}{\Gamma(k)\theta^k}\exp\left\{-\frac{y}{\theta}\right\},\quad y > 0,\quad k, \theta > 0.
\]
Модель гамма распределения выбирается, когда известно, что зависимая переменная принимает только положителньые значения (в отличие от линейной модели, где зависимая переменная принимает любые значения), например, она представляет собой некоторый физический показатель. 

Математическое ожидание и дисперсия равны соответственно $\mathsf E y = \theta/\phi = \mu$ и $\mathsf D y = \mu^2 \phi$, где $\phi=1/k$. Перепараметризуем плотность:
\[
    p(y; \mu, \phi)=\frac{1}{y\Gamma(1/\phi)}\left(\frac{y}{\phi\mu}\right)^{1/\phi} \exp\left\{-\frac{y}{\phi\mu}\right\},\quad y > 0,\quad \mu,\phi>0.
\]
Логарифм функции правдоподобия
\[
    L(\mathbf{y}; \mu, \phi)=-n\ln\Gamma(1/\phi)+\sum_{i=1}^n\left[\frac1\phi\ln\frac{y_i}{\phi\mu} - \ln y_i- \frac{y_i}{\phi\mu} \right].
\]

Поскольку $\mu>0$, в качестве линк-функции фозьмем $g(x)=\ln(x)$. Тогда производные по вектору параметров $\bm\beta$ и $\phi$:
\[
    \frac{\partial L}{\partial \bm\beta}=\frac1\phi\mathbf{X}^\mathrm{T} (\mathbf{y} - \bm\mu) / \bm\mu,\quad \frac{\partial L}{\partial \phi}=\frac1{\phi^2}\left(n \cdot\psi(1 / \phi) - n + \sum_{i=1}^n\left[-\ln\frac{y_i}{\phi\mu_i}+ \frac{y_i}{\mu_i}\right]\right),
\]
где деление векторов производится поэлементно, $\psi(x)=\Gamma^\prime(x)/\Gamma(x)$ — дигамма-функция.

\section{Что было сделано}
Мной был реализован стохастический градиентный спуск (SGD) для обучения обобщенно линейной модели (GLM) на языке Python в двух версиях:
\begin{itemize}
    \item Momentum:
          \begin{align*}
              v_{k+1} & = \eta_1 v_{k} - (1 - \eta_1) \cdot \alpha \nabla f(x_k), \\
              x_{k+1} & = x_k + v_{k+1};
          \end{align*}
    \item Adam (ADAptive Momentum):
          \begin{align*}
              v_{k+1} & = \eta_1 v_{k} + (1 - \eta_1)  \nabla f(x_k),               \\
              G_{k+1} & = \eta_2 G_{k} + (1 - \eta_2) \left(\nabla f(x_k)\right)^2, \\
              x_{k+1} & = x_k - \frac{\alpha}{\sqrt{G_{k+1} + \varepsilon}}v_{k+1}.
          \end{align*}
\end{itemize}
Мной были написаны только варианты для линейной регрессии и GLM с гамма-распределением, но метод обобщается и на другие модели GLM определением соответствующего класса, наследующего класс Family.

\section{Проверка работы метода}
Рассмотрим синтетический и реальный пример и обучим SGD на этих данных. В качестве варианта SGD будем использовать Adam.
\subsection{Проверка на сгенерированных данных}
Пусть количество наблюдений равно $n=1000$, количество признаков (с учетом константного члена) $p=10$. Столбцы $X_i$ матрицы данных $\mathbf{X}=[1:X_1:X_2:\ldots:X_9]$ генерировалась из $\mathrm{U}(0, 1)$, вектор параметров $\bm\beta$ --- из $\mathrm{U}(-10, 10)$. Параметр дисперсии $\phi$ равен $0.01$ в случае линейной регрессии и $0.1$ в случае гамма-регрессии.

Будем смотреть на близость оцененных параметров к истинным. Помимо этого, будем вычислять коэффициент детерминации $R_\text{adj}^2$ в случае линейной модели и его аналог в случае гамма-регрессии. В таблице~\ref{tab:random_data} представлены результаты SGD.

\begin{table}[h]
    \caption{Результат работы SGD на сгенерированных данных}
    \centering
    \subfloat[Линейная регрессия ($R_\text{adj}^2=0.9995$)]{
        \begin{tabular}{|c|>{\centering\arraybackslash}m{0.7in}|>{\centering\arraybackslash}m{0.7in}|}
            \hline
                     & Оцененное значение & Истинное значение \\
            \hline
            const    & 3.935208           & 3.929384          \\
            $X_1$    & -4.254903          & -4.277213         \\
            $X_2$    & -5.470059          & -5.462971         \\
            $X_3$    & 1.028168           & 1.026295          \\
            $X_4$    & 4.379706           & 4.389379          \\
            $X_5$    & -1.544207          & -1.537871         \\
            $X_6$    & 9.620079           & 9.615284          \\
            $X_7$    & 3.711597           & 3.696595          \\
            $X_8$    & -0.383994          & -0.381362         \\
            $X_9$    & -2.180469          & -2.157650         \\
            $X_{10}$ & -3.131106          & -3.136440         \\
            $\phi$   & 0.008969           & 0.01              \\
            \hline
        \end{tabular}
    }\qquad
    \subfloat[Гамма-регрессия (Pseudo $R^2=0.9922$)]{
        \begin{tabular}{|c|>{\centering\arraybackslash}m{0.7in}|>{\centering\arraybackslash}m{0.7in}|}
            \hline
                     & Оцененное значение & Истинное значение \\
            \hline
            const   & 3.941278           & 3.929384          \\
            $X_1$    & -4.263230          & -4.277213         \\
            $X_2$    & -5.452918          & -5.462971         \\
            $X_3$    & 1.000984           & 1.026295          \\
            $X_4$    & 4.324310           & 4.389379          \\
            $X_5$    & -1.535370          & -1.537871         \\
            $X_6$    & 9.645510           & 9.615284          \\
            $X_7$    & 3.712122           & 3.696595          \\
            $X_8$    & -0.379930          & -0.381362         \\
            $X_9$    & -2.164776          & -2.157650         \\
            $X_{10}$ & -3.133490          & -3.136440         \\
            $\phi$   & 0.092371           & 0.1               \\
            \hline
        \end{tabular}
    }
    \label{tab:random_data}
\end{table}

\subsection{Проверка на реальных данных}
В качестве данных рассмотрим датасет, содержащий информацию о сделках с недвижимостью. Он состоит из следующих столбцов:
\begin{itemize}
    \item Transaction date --- дата совершения сделки.
    \item House age --- возраст дома в годах на момент совершения сделки.
    \item Distance to the nearest MRT station --- близость к ближайшей станции общественного скоростного транспорта в метрах.
    \item Number of convenience stores --- количество круглосуточных магазинов поблизости.
    \item Latitude --- широта.
    \item Longitude --- долгота.
    \item House price of unit area --- цена за единицу площади объекта недвижимости.
\end{itemize}
Будем сравнивать самописный метод с методами из библиотек: для линейной регрессии был выбран класс \textsf{SGDRegressor} из библиотеки \textsf{scikit-learn}, для гамма-регресси --- класс \textsf{GLM} из библиотеки \textsf{statsmodels}.

\end{document}